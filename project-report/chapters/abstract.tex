%!TEX root = ../report.tex

\begin{document}
    \begin{abstract}
        Unidirectional tapes with reinforced fibers such as carbon, natural or glass have incredible load bearing limits, particulary due to their complex and hetrogenous micro-structure .These tapes have superior strength to weight ratio, providing greater resistance to corrosive chemicals such as destructive synthetic compounds, prevent water absorption and heat. These tapes are affected by the manufacturing process in the production of semi finished product and in completed auxiliary segments.For example each process in the production stage contributes towards void in the tapes and fiber orientations in the composite. As a result the techniques that help to understand the properties of these tapes are known as non-destructive evaluation(NDE). These include eddy current sensing, ultra sonic sensing, thermal scanning, micro-CT scanning and much more. With the digital transformation under "Industry 4.0" or "IIOT", we have a digital twin that presents a representation of the physical entity performed by NDE techniques. Using this, a structural change can be monitored in an data driven quality assurance control system. With much focus on artificial intelligence and learning algorithms , the aim of this research is to leverage this digital twin of a unidirectional tape, make use of intelligence to comprehend and reason the quality of the final product itself. This research shows an effective characterising of features that contribute towards the quality of the reinforced unidirectional tape in relation to the number of porosities and dents present in a test specimen. Shape, area, and size features extracted from the test specimen are carried out using image processing. As a part of the production process, due to vibrations occurring in the manufacturing stage, a region of interest selection for finding plausible feautures is to be deduced. Thus finally using stastical analysis and also a classifier based neural network showing the appropriate region of interest to be utilised inorder to obtain features from the test specimen is proposed.  
    \end{abstract}
\end{document}
