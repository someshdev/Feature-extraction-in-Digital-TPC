%!TEX root = ../report.tex

\begin{document}
    \chapter{Introduction}
Unidirectional tapes (UD tapes), are thermoplastic composite comprised of reinforced fibers usually of the type glass, carbon or natural fibers. Thermoplastic unidirectional tapes[1] prove increased reinforcement when it comes to composite markets. Today with numerous applications for light weight series production, for a thermoplastic composite and the endless fiber reinforced sheets an evolution of a more mature supply chain has begun, which is very important if not crucial the adoption of TPC in general. In particular to their micro structure of the UD tapes, they are stronger, lighter and smoother, resulting towards non-crimp fabrics. The properties of a composite, promise an alternative option in contrast to woven  fiber reinforced sheets. These fibers can be aligned into directions specific to the load requirement. As a result these tapes provide minimum scap rate and have decreased production cycle time[2].  Also the smooth surfaces provide better aesthetics. Every step in the production of the UD tape is mass scalable, that is from a UD tape to a final product. The production of these tapes are influenced by manufacturing processes , providing an enormous challenge to process controls such as quality assurance[3].

	The general pipeline for a composite production, starts with tape creels with tensioning devices, feeded by precise cutting units and a lay up system.To achieve a cross view in understanding a real and a virtual process, digital twins play an significant role. As a result, the concept of real time digital twin leads towards individualized features analyzed and optimizing the response to the input variable resulting in an optimum end value product[4]. The digital twin for a UD tape can help us understand the debilitating defects, for example local fiber deviations, pores and so on. Thus with the help of sensory augmented data, we can have a visualization representing the real world events and physical characteristics of structures and processes. The techniques that help examine the quality of a UD tapes are known as non destructive evaluation techniques. These techniques include eddy current sensing, ultra sonic sensing, thermal scanning, micro-CT scanning and much more. With these, a structural change can be monitored in an data driven quality assurance control system. As a result there is no single method to generalize sufficient enough to detect all errors safely. The difference in the fiber deliminations, porosities, dents and degradation of the fiber contribute towards the quality of the UD tape .This provides a challenge in passing the data between different disciplines. Accordingly data from different domains can be correlated with corresponding interfaces and understood. Using this domain knowledge, we can predict material grades and accordingly adjust parameters along the supply chain. 
	
	According to demanding monitoring strategies during each production stage, process monitoring and image segmentation play a key job for quality evaluation. The following describes  the important aspects.[5]
	\begin{itemize}
	\item Visual monitoring system: Detecting faults and other abnormalities are needed to be deployed, as a result we need to have regular visual inspection. This requires an optical unit and a sensing device for different ambient conditions. Overall the optical unit have cameras for autonomous visual inspection.
	\item Digital image processing: Image enhancement and region of interest are two main primary concerns for acquisition of an image. Some of these process include image restoration, compression, morphological processing, enhancement and region of interest detection. These pipeline have an empowering effect on memory consumption and computational costs. Also understanding the foreground and the background plays the key role for identification of our visual area.
	\item Feature descriptors:
	\end{itemize}


    \section{Motivation}
    \subsection{...}


    \subsection{...}


    \section{Challenges and Difficulties}
    \subsection{...}


    \subsection{...}

    \subsection{...}



    \section{Problem Statement}
    \subsection{...}


    \subsection{...}


    \subsection{...}
\end{document}
